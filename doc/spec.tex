\documentclass{book}

\usepackage{stmaryrd}
\usepackage{amsmath}


\newcommand{\todo}[1]{\underline{TODO: {#1}}}
\newcommand{\sem}[1]{\llbracket{#1}\rrbracket}
\newcommand{\evalE}[1]{E_\mathrm{e}\left({#1}\right)}
\newcommand{\evalS}[1]{E_\mathrm{s}\left({#1}\right)}

\newcommand{\expressionsentence}[1]{\mathsf{void}={#1}\mathsf{;}}

\title{Bamboo Specification---An Early Draft}

\begin{document}

\maketitle

This specification contains some exercises.  Please try to solve them at least mentally.  Your solutions are welcome at \texttt{https://gitter.im/bbo-dev/Lobby}.  \todo{use url}

\chapter{Preliminaries}

\section{What Qualifies a Bamboo Implementation}

\ldots A Bamboo compiler can refuse to compile certain valid programs.

\section{Notations}

$v \in A$ says $v$ is an element of a set~$A$.

\chapter{Syntax}

\section{Keywords}

\newcommand{\abort}{\text{\texttt{abort;}}}
\newcommand{\true}{\text{\texttt{true}}}
\newcommand{\false}{\text{\texttt{false}}}
\newcommand{\msgsender}{\text{\texttt{msg.sender}}}
\newcommand{\msgvalue}{\text{\texttt{msg.value}}}
\newcommand{\this}{\text{\texttt{this}}}
\newcommand{\now}{\text{\texttt{now}}}
\newcommand{\paren}[1]{\mathtt{(}{#1}\mathtt{)}}
\newcommand{\logicalAnd}[2]{{#1}\mathbin{\text{\texttt{\&\&}}{#2}}}
\newcommand{\logicalNot}[1]{\mathop{\text{\texttt{not}}}{#1}}
\newcommand{\balance}[1]{\text{\texttt{balance}}\mathtt{(}{#1}\mathtt{)}}
\newcommand{\arrayAccess}[2]{{#1}[{#2}]}
\newcommand{\lt}[2]{{#1} \mathop{\text{\texttt{<}}} {#2}}
\newcommand{\gt}[2]{{#1} \mathop{\text{\texttt{>}}} {#2}}
\newcommand{\eq}[2]{{#1} \mathop{\text{\texttt{==}}} {#2}}
\newcommand{\notEq}[2]{{#1} \mathop{\text{\texttt{!=}}} {#2}}

The following sequences of characters are \textit{keywords}.
\begin{itemize}
\item $\true$
\item $\false$
\item $\this$
\item $\now$
\item \texttt{not}
\item \texttt{contract}
\item \texttt{default}
\item \texttt{case}
\item \texttt{abort}
\item \texttt{uint8}
\item \texttt{uint256}
\item \texttt{bytes32}
\item \texttt{address}
\item \texttt{bool}
\item \texttt{if}
\item \texttt{else}
\item \texttt{then}
\item \texttt{become}
\item \texttt{return}
\item \texttt{deploy}
\item \texttt{with}
\item \texttt{reentrance}
\item \texttt{selfdestruct}
\item \texttt{block}
\item \texttt{void}
\item \texttt{event}
\item \texttt{log}
\item \texttt{indexed}
\end{itemize}

\todo{Use a maththm like environment for exercises.}

Exercise: which of the following are keywords?
\begin{enumerate}
\item \texttt{True}
\item \texttt{true}
\end{enumerate}

\section{Identifier}

An \textit{identifier} is a sequence of characters that matches the following regular expression (but is not a keyword):
\begin{verbatim}
  ['a'-'z' 'A'-'Z' '_'] ['a'-'z' 'A'-'Z' '0'-'9' '_']*
\end{verbatim}

\todo{define identifier}


Exercise: which of the following are identifiers? \\
\todo{complete}

\section{Syntactic Types}

The following sequences of characters are \textit{syntactic types}.
\begin{itemize}
\item \texttt{void}
\item \texttt{uint256}
\item \texttt{bool}
\item \texttt{uint8}
\item \texttt{bytes32}
\item \texttt{address}
\item \texttt{mapping } $T$ \texttt{=>} $T'$ when $T$ and $T'$ are syntactic types
\item any identifier except those listed above.
\end{itemize}

Every syntactic type except \texttt{void} is a \textit{non-\texttt{void} syntactic type}.

Exercise: which of the following is true?
\begin{enumerate}
\item \texttt{uint256} is the set of integers at least zero and at most $2^{256} - 1$.
\item \texttt{uint256} is the set of 256-bit words.
\item none of the above.
\end{enumerate}

\section{Expressions}

A \textit{case-call-expression} looks like
\[
e . c (e_1, \ldots, e_n)\ m
\]
with $e$ and $e_i$ ($1 \le i \le n$) being expressions and $m$ a message-info.

A \textit{call-expression} is either a case-call-expression or a default-call-expression.

A \textit{deploy-expression} looks
\[
\text{\texttt{deploy}}\ i(e_1,\ldots,e_n)\ m
\]
with $i$ being an identifier, $e_i$ ($1 \le i \le n$) an expression and $m$ a message-info.

\textit{Expressions} are inductively defined as follows.

\begin{itemize}
\item $\true$ is an expression.
\item $\false$ is an expression.
\item $\msgsender$ is an expression.
\item $\msgvalue$ is an expression.
\item $\this$ is an expression.
\item $\now$ is an expression.
\item An identifier is an expression.
\item When $e$ is an expression, $\paren{e}$ is an expression.
\item When $e$ is an expression, $\logicalNot{e}$ is an expression.
\item When $e$ is an expression, $\balance{e}$ is an expression.
\item A deploy-expression is an expression.
\item A call-expression is an expression.
\item AddressExp?  What is an AddressExp?
\item When $e_0$ and $e_1$ are expressions, the following are expressions
\begin{itemize}
\item $\logicalAnd{e_0}{e_1}$
\item $\lt{e_0}{e_1}$
\item $\gt{e_0}{e_1}$
\item $\notEq{e_0}{e_1}$
\item $\eq{e_0}{e_1}$
\item $\arrayAccess{e_0}{e_1}$
\item $e_0 + e_1$
\item $e_0 - e_1$
\item $e_0 \times e_1$
\end{itemize}
\end{itemize}

Exercise: prove that \texttt{99a} is not an expression.

\section{Sentences}

A \textit{return sentence} looks
\[
\texttt{return}\, e_0 \,\texttt{then}\, \texttt{become}\, i(e_1,\ldots,e_n);
\]
with $i$ being an identifier and $e_i$ ($0 \le i \le n$) an expression.

An \textit{assignment sentence} looks
\[
l = e;
\]
with $l$ being a left-expression and $e$ an expression.

A \textit{variable initialization sentence} looks
\[
t i = e;
\]
with $t$ being a type, $i$ and identifier and $e$ an expression.

\textit{Sentences} are inductively defined as follows.

\begin{itemize}
\item $\abort$ is a sentence.
\item a return sentence is a sentence.
\item an assignment sentnece is a sentence.
\item a variable initialization sentence is a sentence.
\item an expression-only sentence is a sentence.
\item an if sentence is a sentence.
\item an if-then-else sentence is a sentence.
\item a logging sentence is a sentence.
\item a self-destruct sentence is a sentence.
\end{itemize}

\section{Blocks}

A \textit{block} is a possibly empty sequence of stentences surrounded by \texttt{\{} and \texttt{\}}.

\todo{talk about whitespaces, perhaps.}

\section{Cases}

A \textit{case} is a case header followed by a block.


\section{Contract Headers}

\section{Contracts}

A \textit{contract} is a contract header followed by a \texttt{\{}, some (possibly no) cases and a \texttt{\}}.

\subsection{Contract's Signature}

$(\mathtt{auction}, [\mathtt{address}, \mathtt{uint256}, \mathtt{address}, \mathtt{uint256}])$.

Exercise: what is the signature of the following contract?
\todo{complete the question}

\subsection{Contract Body}

A \textit{contract body} is a possibly empty sequence of cases surrounded by \texttt{\{} and \texttt{\}}.

\subsection{Contract Definitions}

A \textit{contract definition} is a contract header followed by a contract body.

\section{Event Declarations}

\chapter{Semantics}

\section{Notations}

\todo{describe $\in$}

\todo{describe pi}

\section{States}

\subsection{Values}

A \textit{value} is a 256-bit word.

We pick something called $\bot$ (pronounced ``bottom'') which is not a value.  The choice should not affect the meaning of a Bamboo program.

(We might specify a set of values for each syntactic type in the future.)

\subsection{A Contract's States}
A contract has \textit{states}.

When a contract has a signature $(x, [T_1, T_2, \ldots, T_n])$ ($n \ge 0$),
the set of the states of the contract is
$\Pi_{i = 0}^{n} \sem{T_i}$.

\subsection{A Program's Account States}
A program determines a set of \textit{account states}.

\section{Dynamics}

\subsection{Variable Environment}

A \textit{variable environment} is a partial map that takes identifiers and may or may not return a value.  When a variable environment $\sigma$ maps and identifier~$i$ to a value~$v$, we write
\[
\sigma(i) = v
\]

\subsection{Current Call}

A \textit{current call} $c = (c_\mathrm{s}, c_\mathrm{v}, c_\mathrm{t})$ is a tuple of
\begin{itemize}
\item a value $c_\mathrm{s}$ called the \textit{sender},
\item a value $c_\mathrm{v}$ called the \textit{transferred amount} and
\item a value $c_\mathrm{t}$ called the \textit{timestamp}.
\end{itemize}

\subsection{World Oracle}

\subsubsection{Call Queries}

\subsubsection{Create Queries}

\subsubsection{Balance Queries}

\subsection{Timestamp query}

\newcommand{\timestampQuery}{\text{\texttt{timestamp?}}}
The time stamp query~$\timestampQuery$ is something different from any query that appears above. Apart from that, $\timestampQuery$ can be anything, and the behavior of Bamboo programs should not be affected by the concrete choice of $\timestampQuery$ (with some adaptations on world oracles).

\subsection{Current Account Query}

\newcommand{\thisQuery}[0]{\text{\texttt{this?}}}

The current account query~$\thisQuery$ is something different from any query that appears above.

\subsection{Sender Query}

\newcommand{\senderQuery}{\text{\texttt{sender?}}}

The sender query~$\senderQuery$ is something different from any query that appears above.

\subsection{Value Query}

\newcommand{\valueQuery}{\text{\texttt{value?}}}

The value query~$\valueQuery$ is something different from any query that appears above.

\subsubsection{World Oracle}

Call queries, create queries, \todo{fill in} are \textit{oracle queries}.

A \textit{world oracle} is defined coinductively as a function that takes an oracle query and returns a pair of a value and a world oracle.

When a world oracle~$w$ takes a query~$q$ and returns a pair of a value~$v$ and a world oracle~$w'$, we write $w(q) = (v, w')$.

\todo{Add a possibility that the world oracle calls into the program again.}

\subsection{Evaluation of an Expression}

The evaluation for expressions takes
\begin{itemize}
\item an expression,
\item a current call,
\item a world oracle and
\item a variable environment.
\end{itemize}
It returns
\begin{itemize}
\item a value or $\bot$ and
\item a world oracle
\end{itemize}



\todo{say it's inductively defined over the definition of expressions}.

\subsubsection{Evaluation of Literals}

\textit{Literals} are those keywords whose evaluation is defined below.

\begin{itemize}
  \item $\evalE{\boxed{\true}, c, w, \sigma} := (1, w)$
  \item $\evalE{\boxed{\false}, c, w, \sigma} := (0, w)$
  \item $\evalE{\boxed{\now}, c, w, \sigma} := w(\timestampQuery)$
  \item $\evalE{\boxed{\this}, c, w, \sigma} := w(\thisQuery)$
  \item $\evalE{\boxed{\msgsender}, c, w, \sigma} := w(\senderQuery)$
  \item $\evalE{\boxed{\msgvalue}, c, w, \sigma} := w(\valueQuery)$
\end{itemize}

balance(x) sends a balance query on the world oracle.

\subsubsection{Evaluation of an Identifier}

An identifier~$i$ is evaluated as follows:
\[
\evalE{\boxed{i}, c, w, \sigma} := (\sigma(i), w)
\]

Note that $\sigma(i)$ can be $\bot$.

\subsubsection{Evaluation of a new-expression}

\todo{Consider new-expressions with nontrivial continuation later.  That requires an interaction between the program and the world oracle. }

\subsubsection{Evaluation of a call-expression}

\todo{Consider call-expressions with nontrivial continuation later.  That requires an interaction between the program and the world oracle. }

\subsubsection{Evaluation of Binary Operators}

\[
\evalE{\boxed{\logicalAnd{e_0}{e_1}}, c, w, \sigma} :=
\begin{cases}
  (v_0, w_0) & \text{if}\ v_0 = 0\ \text{or}\ v_0 = \bot \\
  (v_1, w_1) & \text{otherwise}
\end{cases}
\]
where
\[
\evalE{\boxed{e_0}, c, w, \sigma} = (v_0, w_0)
\]
and
\[
\evalE{\boxed{e_1}, c, w_0, \sigma} = (v_1, w_1)
\]

\[
\evalE{\boxed{\lt{e_0}{e_1}}, c, w, \sigma} :=
\begin{cases}
  (1, w_0) &\text{if}\ v_0 \neq \bot, v_1 \neq\bot\ \text{and}\ v_0 < v_1 \\
  (0, w_0) &\text{if}\ v_0 \neq \bot, v_1 \neq\bot\ \text{and}\ v_0 \ge v_1 \\
  (\bot, w) &\text{if}\ v_0 = \bot \ \text{or}\ v_1 = \bot
\end{cases}
\]
where
\[
\evalE{\boxed{e_1}, c, w, \sigma} = (v_1, w_1)
\]
and
\[
\evalE{\boxed{e_0}, c, w, \sigma} = (v_0, w_0)
\]

\[
\evalE{\boxed{\arrayAccess{e_0}{e_1}}, c, w, \sigma, M} :=
(w_0, M(v_0, v_1))
\]
where
\[
\evalE{\boxed{e_1}, c, w, \sigma, M} = (v_1, w_1)
\]
and
\[
\evalE{\boxed{e_0}, c, w_1, \sigma, M} = (v_0, w_0)
\]

\subsection{Evaluation of a Sentence}

The evaluation function for sentences take
\begin{itemize}
\item a sentence,
\item a current call,
\item a variable environment and
\item a world oracle
\end{itemize}
and returns
\begin{itemize}
\item a variable environment,
\item a world oracle
\item and optionally an account state.
\end{itemize}

\todo{Show two forms of equations: one without an account state, the other with an account state.}

\subsubsection{Evaluation of an Expression Sentence}

\[
\evalS{\boxed{\expressionsentence{e}}, c, \sigma, w} := (\sigma', w')
\]
where
\[
\evalE{\boxed{e}, c, \sigma, w} = (v, \sigma', w')
\]

\subsection{Evaluation of a Case}

The evaluation function of a case takes
\begin{itemize}
\item A contract state
\item a world oracle
\item a case call
\end{itemize}
and returns
\begin{itemize}
\item an account state
\item a world oracle
\end{itemize}

\subsection{Evaluation of a Contract}

The evaluation function of a contract takes
\begin{itemize}
\item A contract state
\item a world oracle
\item a contract call
\end{itemize}
and returns
\begin{itemize}
\item An account state
\item a world oracle
\end{itemize}

\subsection{Evaluation of a Program}

The evaluation function of a program takes
\begin{itemize}
\item an account state
\item a world oracle
\item a program call
\end{itemize}
and returns
\begin{itemize}
\item an account state
\item a world oracle
\end{itemize}

\section{Account Initialization}

\subsection{Account Deployment Query}

\subsection{Initial Variable Environment}

\chapter{Connection to EVM}

\section{Bamboo Account State as an EVM Account State}

\section{Queries as EVM instructions}

\end{document}
